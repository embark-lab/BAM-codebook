% Options for packages loaded elsewhere
\PassOptionsToPackage{unicode}{hyperref}
\PassOptionsToPackage{hyphens}{url}
%
\documentclass[
]{article}
\usepackage{amsmath,amssymb}
\usepackage{lmodern}
\usepackage{iftex}
\ifPDFTeX
  \usepackage[T1]{fontenc}
  \usepackage[utf8]{inputenc}
  \usepackage{textcomp} % provide euro and other symbols
\else % if luatex or xetex
  \usepackage{unicode-math}
  \defaultfontfeatures{Scale=MatchLowercase}
  \defaultfontfeatures[\rmfamily]{Ligatures=TeX,Scale=1}
\fi
% Use upquote if available, for straight quotes in verbatim environments
\IfFileExists{upquote.sty}{\usepackage{upquote}}{}
\IfFileExists{microtype.sty}{% use microtype if available
  \usepackage[]{microtype}
  \UseMicrotypeSet[protrusion]{basicmath} % disable protrusion for tt fonts
}{}
\makeatletter
\@ifundefined{KOMAClassName}{% if non-KOMA class
  \IfFileExists{parskip.sty}{%
    \usepackage{parskip}
  }{% else
    \setlength{\parindent}{0pt}
    \setlength{\parskip}{6pt plus 2pt minus 1pt}}
}{% if KOMA class
  \KOMAoptions{parskip=half}}
\makeatother
\usepackage{xcolor}
\IfFileExists{xurl.sty}{\usepackage{xurl}}{} % add URL line breaks if available
\IfFileExists{bookmark.sty}{\usepackage{bookmark}}{\usepackage{hyperref}}
\hypersetup{
  pdftitle={Body Advocacy Movement - Study Procedures and Measures},
  pdfauthor={Katherine Schaumberg},
  hidelinks,
  pdfcreator={LaTeX via pandoc}}
\urlstyle{same} % disable monospaced font for URLs
\usepackage[margin=1in]{geometry}
\usepackage{longtable,booktabs,array}
\usepackage{calc} % for calculating minipage widths
% Correct order of tables after \paragraph or \subparagraph
\usepackage{etoolbox}
\makeatletter
\patchcmd\longtable{\par}{\if@noskipsec\mbox{}\fi\par}{}{}
\makeatother
% Allow footnotes in longtable head/foot
\IfFileExists{footnotehyper.sty}{\usepackage{footnotehyper}}{\usepackage{footnote}}
\makesavenoteenv{longtable}
\usepackage{graphicx}
\makeatletter
\def\maxwidth{\ifdim\Gin@nat@width>\linewidth\linewidth\else\Gin@nat@width\fi}
\def\maxheight{\ifdim\Gin@nat@height>\textheight\textheight\else\Gin@nat@height\fi}
\makeatother
% Scale images if necessary, so that they will not overflow the page
% margins by default, and it is still possible to overwrite the defaults
% using explicit options in \includegraphics[width, height, ...]{}
\setkeys{Gin}{width=\maxwidth,height=\maxheight,keepaspectratio}
% Set default figure placement to htbp
\makeatletter
\def\fps@figure{htbp}
\makeatother
\setlength{\emergencystretch}{3em} % prevent overfull lines
\providecommand{\tightlist}{%
  \setlength{\itemsep}{0pt}\setlength{\parskip}{0pt}}
\setcounter{secnumdepth}{5}
\usepackage{booktabs}
\usepackage{amsthm}
\makeatletter
\def\thm@space@setup{%
  \thm@preskip=8pt plus 2pt minus 4pt
  \thm@postskip=\thm@preskip
}
\makeatother
\ifLuaTeX
  \usepackage{selnolig}  % disable illegal ligatures
\fi
\usepackage[]{natbib}
\bibliographystyle{plainnat}

\title{Body Advocacy Movement - Study Procedures and Measures}
\author{Katherine Schaumberg}
\date{2022-09-05}

\begin{document}
\maketitle

{
\setcounter{tocdepth}{2}
\tableofcontents
}
\hypertarget{study-premise}{%
\section{Study Premise}\label{study-premise}}

The Body Advocacy Movement Study (Phase 1: 2021-2023) is a pilot study investigating the feasiblility, acceptability, and preliminary efficacy of an eating disorder risk reduction intervention developed by the EMBARK lab in 2020.

While existing, dissonance programs (i.e.~the Body Project) target reductions in thin-ideal internalization, fears related to fatness and weight gain may be a particularly salient cognitive target among those with current or past eating disorder symptomatology \citep{levinsonAddressingFearFat2014}. In an effort to mitigate risk for ED onset and relapse among high-risk young adults (i.e.~those in ED partial- or full-remission and/or those with ongoing subthreshold ED symptoms), the current project includes development and initial evaluation of a novel, peer-led intervention that focuses specifically on fear of fatness and anti-fat bias (the Body Advocacy Movement; BAM).

This project aims to recruit 100 young adults (ages 18-26) -- randomized to a standard dissonance intervention program (the Body Project; N = 33) or BAM (N = 66). Our objective is to examine the degree to which BAM can reduce fear of fatness and weight gain along with internalized anti-fat bias and reduce risk for ED symptom exacerbation and promote ongoing ED recovery in a high-risk, young adult sample. We will evaluate feasibility and acceptability of BAM and estimate effects of the intervention on self-report and behavioral measures of fear of weight gain and fatness, anti-fat bias, and eating disorder symptoms.

\hypertarget{specific-aims}{%
\subsection{Specific Aims}\label{specific-aims}}

\textbf{Aim 1}: Examine the acceptability and feasibility of BAM in a high-risk young adult sample. Acceptability: Acceptability will be assessed via attrition across a 4-session intervention along with qualitative feedback from a brief survey at post-intervention. We expect to meet benchmarks of high acceptability (average ratings of \textgreater{} 4 {[}out of 5{]} on post-intervention survey; \textless{} 20\% attrition). Feasibility: Feasibility will be benchmarked via intervention adherence (average competency \textgreater{} 8.0 on 0-10 scale) and recruitment goals.

\textbf{Aim 2}: Estimate the effects of the intervention on fear of fatness, anti-fat bias, and eating disorder symptoms. We hypothesize that the BAM intervention will result in acute reductions (post-intervention) in fatphobia and anti-fat bias across self---report and behavioral measures, outperforming the Body Project on targets related to anti-fat bias, specifically.

\hypertarget{procedures}{%
\section{Procedures}\label{procedures}}

Individuals are recruited to BAM via \ldots.xxx

\hypertarget{interventions}{%
\section{Interventions}\label{interventions}}

\hypertarget{the-body-project}{%
\subsection{The Body Project}\label{the-body-project}}

\hypertarget{the-body-advocacy-movement}{%
\subsection{The Body Advocacy Movement}\label{the-body-advocacy-movement}}

\hypertarget{self-report-measures}{%
\section{Self-Report Measures}\label{self-report-measures}}

\hypertarget{task-based-measures}{%
\section{Task-based Measures}\label{task-based-measures}}

here are some task based measures

\hypertarget{sample-description}{%
\section{Sample Description}\label{sample-description}}

Individduals were recruited to the Body Advocacy Movement through a student organizaiton developed simultanously at the University of Wisconsin - Madison

\hypertarget{data-requests}{%
\section{Data Requests}\label{data-requests}}

  \bibliography{BAM-bookdown.bib}

\end{document}
